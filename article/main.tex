\documentclass[a4paper]{article}

%% Language and font encodings
\usepackage[english]{babel}
\usepackage[utf8]{inputenc}
\usepackage{times}

%% Sets page size and margins
\usepackage[a4paper,top=3cm,bottom=2cm,left=3cm,right=3cm,marginparwidth=1.75cm]{geometry}

%% Useful packages
\usepackage{amsmath}
\usepackage{graphicx}
\usepackage{xcolor}

%% multiline equations
\usepackage{amsmath}

%% line number
\usepackage[left]{lineno}
\linenumbers

%% double space (with \doublespacing)
\usepackage{setspace}
\doublespacing

%% citation / bibliography
\usepackage[round,authoryear]{natbib}


\usepackage{authblk}

\title{PROTEST}

\author[1]{Raphaël Aussenac}
\author[1]{Patrick Vallet}
\author[1]{Jean-Matthieu Monnet}
\author[2]{Thomas Carrette}
\author[3]{Pierre Paccard}
\author[4]{Arnaud Sergent}
\author[5]{Catherine Riond}
\author[6]{Raphaël Bondu}
\author[1]{Vanneck Nzeta Kenne}

\affil[1]{IRSTEA grenoble}
\affil[2]{FCBA grenoble}
\affil[3]{PNR Bauges}
\affil[4]{IRSTEA Bordeaux}
\affil[5]{ONF}
\affil[6]{???}


\date{}

\begin{document}
\maketitle

\section*{Abstract}

\section*{Journal}
\begin{itemize}
    \item Environmental Modelling \& Software (suggestion Patrick)
    \item Agricultural and Forest Meteorology (Coupling transversal and longitudinal models to better predict Quercus petraea and Pinus sylvestris stand growth under climate change)
    \item Forest Ecology and Management
    \item Annals of Forest Science (Capsis: an open software framework and communityfor forest growth modelling)
    \item Journal of Environmental Management
    \item Methods in ecology and evolution (papier sur méthode d'initialisation)
\end{itemize}

%%%%%%%%%%%%%%%%%%%%%%%%%%%%%%%%%%%%%%%%%%%%%%%%%%%%%%%%%%%%%%%%%%%%%%%%%%%%%%%%%
\section*{Introduction}

\citep{cardinale2012biodiversity}

%%%%%%%%%%%%%%%%%%%%%%%%%%%%%%%%%%%%%%%%%%%%%%%%%%%%%%%%%%%%%%%%%%%%%%%%%%%%%%%%%
\section*{Materials and Methods}

%%%%%%%%%%%%%%%%%%%%%%%%%%%%%%%

\subsection*{Geological classification - Raphaël Bondu}

Description de la variable “code carbonate”:

Cette variable décrit la teneur en minéraux carbonatés de la roche-mère suivant 4 catégories :
0  Pas à peu de carbonates (ex. schiste)
1  Présence modérée de carbonates (ex. marne)
2  Riche en carbonates (ex. calcaire argileux)
3  Carbonates quasi-purs (ex. calcaire lithographique)
Lorsque les ensembles géologiques regroupent des formations variées du point de vue lithologique, la valeur assignée a tenté de tenir compte de l’épaisseur de chacune des formations.
Cette classification est basée sur les informations qualitatives tirées des documents du Bureau de Recherche Géologique et Minière (BRGM) suivants :
Doudoux B., Barféty J.C., Vivier G., Carfantan J.C., Nicoud G., Tardy M., avec la collaboration de Monjuvent G., Debon F., Menot R.-P., Aprahamian J. (1999) - Notice explicative, Carte géol. France (1/50 000), feuille Albertville (726). Orléans : BRGM, 119 p. Carte géologique par B. Doudoux, J.-C. Barféty, G. Vivier, J.-C. Carfantan, G. Nicoud, B. Colletta, M. Tardy (1999).
F. Cagnard (2008) – Carte géologique harmonisée du département de la Haute-Savoie. BRGM/RP-5 59 30 -FR, 329 p.,7 fig., 2 tab., 3 pl. Hors-texte.

\noindent Description de la variable “code hydro”:

Cette variable décrit la perméabilité potentielle de la roche-mère en fonction de sa nature minéralogique suivant 4 catégories :
0  Fortement perméable (ex. calcaire massif karstifié)
1  Perméable (ex. calcaire argileux)
2  Peu perméable (ex. alternance marne et calcaire)
3  Imperméable ou quasi-imperméable (ex. argile)
Cette classification est basée sur les informations qualitatives tirées des documents du Bureau de Recherche Géologique et Minière (BRGM) suivants :
Doudoux B., Barféty J.C., Vivier G., Carfantan J.C., Nicoud G., Tardy M., avec la collaboration de Monjuvent G., Debon F., Menot R.-P., Aprahamian J. (1999) - Notice explicative, Carte géol. France (1/50 000), feuille Albertville (726). Orléans : BRGM, 119 p. Carte géologique par B. Doudoux, J.-C. Barféty, G. Vivier, J.-C. Carfantan, G. Nicoud, B. Colletta, M. Tardy (1999).
F. Cagnard (2008) – Carte géologique harmonisée du département de la Haute-Savoie. BRGM/RP-5 59 30 -FR, 329 p.,7 fig., 2 tab., 3 pl. hors-texte.

%%%%%%%%%%%%%%%%%%%%%%%%%%%%%%%

\subsection*{Landscape initialisation}
To run SIMMEM simulations, the following variables must be available for each forest plot:
\begin{itemize}
    \item site index
    \item composition
    \item species basal area
    \item number of trees per species (species quadratic diameter must be calculated for that)
\end{itemize}

%%%%%%%%%%%%%%%%%%%%%%%%%%%%%%%

\subsection*{Site index}

\noindent We assigned a species-specific site index ($si$) to each forest plot over the study area following a two-step procedure:

\begin{enumerate}
    \item Because not all the variables required to calculate $si$ were available at all locations over the study area, we first modeled the available $si$ at the NFI plots $i$ from available environmental variables and LIDAR-derived measures, using Generalized linear models with a Gamma ($\Gamma$) probability distribution. The four species-specific models are based on the following initial structure:

    \begin{multline}\label{si}
      \widehat{si_i} = a + b_1 alti_i + b_2 slope_i + b_3 rum_i + b_4 pH_i + b_5 expoNS_i + b_6 expoEW_i + \\ b_7 greco_i + b_8 hyrdo_i + b_9 carbo_i
    \end{multline}

    \begin{equation}\label{si}
      si_i \sim \Gamma(\alpha, \frac{\alpha}{\widehat{si_i}})
    \end{equation}

    where $a$, $b_{1-9}$ and $\alpha$ are parameters to estmiate. Site index being species-specific, we ran four versions of this model, one for each target-species. We then applied to the models the following variable selection procedure:

    \begin{enumerate}

        \item We first removed (one by one) the variables whose effect was non-significant starting with the least significant ones untill all variables were significant.

        \item We then added a quadratic effect to all remaining variables.

        \item Here again, we removed (one by one) the variables whose effect was non-significant starting with the least significant ones untill all variables were significant (see the four final species-specific models in SI).

  \end{enumerate}

    \item We then assigned a $si$ to each forest plot $p$ over the study area by drawing a random number from the Gamma distribution $\Gamma(\alpha, \frac{\alpha}{\widehat{si_p}})$, $\widehat{si_p}$ being the $si$ predicted by the models at plot $p$.

\end{enumerate}

%%%%%%%%%%%%%%%%%%%%%%%%%%%%%%%

\subsection*{Composition}

We assigned a composition to all forest plots over the study area following a four-step procedure:

\begin{enumerate}

    \item First we classified the PROTEST plots into 7 compositions (corresponding to the compositions that will be simulated): 4 pure compositions (beech, oak, fir, spruce; when these species represented more than 75\% of the plots total basal area) and 3 mixed compositions (beech-fir, beech-spruce, fir-spruce; when these species were the 2 most abundant on the plots and represented (together) more than 75\% of the plots total basal area). However, because other species than our 4 target species were present on the study area (ref composition / species abundance?), we previously grouped some species together as follows:

    \begin{enumerate}

        \item When non-target coniferous species (i.e. species other than fir and spruce) were present on a PROTEST plot, their basal area were assigned to fir and/or spruce (depending on the relative proportion of the basal area of these two species) if they were present on the plot.

        \item Similarly, when non-target deciduous species (i.e. species other than oak and beech) were present on a PROTEST plot, their basal area were assigned to oak and/or beech (depending on the relative proportion of the basal area of these two species) if they were present on the plot. However, there is one exception here: the basal area of \textit{Quercus robur} and unspecified oaks have always been assigned to \textit{Quercus petraea}.

    \end{enumerate}

    Here, saplings were taken into account. Viable deciduous saplings and sprouts were considered as unidentified deciduous. Their basal area could therefore be assigned to oak and/or beech afterwards. Viable coniferous saplings were considered as unidentified coniferous. Their basal area could therefore be assigned to fir and/or spruce afterwards. Since we had no information on the identity of the non-viable saplings, we assigned their basal area to deciduous and coniferous saplings depending on the relative proportion of these two categories. Plots whose composition was not among the 7 expected compositions were dropped at this stage.

    \item We then classified all the forest plots over the study area into 3 composition types: deciduous, coniferous or deciduous-coniferous mixture. For that we used the proportion of deciduous basal area ($P_{Gd}$) calculated from LIDAR data. Plots in which the deciduous basal area accounted for more than 75\% of the total basal area were classified as deciduous. Conversely, plots in which the deciduous basal area accounted for less than 25\% of the total basal area were classified as coniferous. The remaining plots were classified as deciduous-coniferous mixture.

    \item Thereafter, we retrieved the list of compositions of the PROTEST plots within each vegetation type (i.e. TFV type). This list could contain repetitions (because the same composition could be observed several times within a single TFV type). However, because 5 TFV types represented more than 96\% of the forest area and because the remaining 3\% belonged to 15 different TFV types, we previously grouped some TFV types together. The 15 under-represented TFV types have either been removed or integrated into one of the 5 main TFV types (see SI).

    \item Finally, to assign a composition to a forest plot located in a specific TFV type, we randomly selected a composition from the list of compositions associated to this TFV type. We made sure that the selected composition was in line with the composition type allready assigned to the plot (i.e. deciduous, coniferous or deciduous-coniferous mixture). Another composition was randomly selected until this condition was met.

\end{enumerate}

%%%%%%%%%%%%%%%%%%%%%%%%%%%%%%%

\subsection*{Species basal area}

\noindent We assigned species basal area values ($G_{sp}$) to each forest plot using the stand total basal area ($G_t$) and the proportion of deciduous basal area ($P_{Gd}$) calculated from LIDAR data. $G_{sp}$ could directly be derived from these measures for pure stands and deciduous-coniferous mixtures (i.e. beech-fir and beech-spruce stands). We developed a two-step procedure to assign $G_{sp}$ to conifer mixtures (i.e. fir-spruce mixtures):

\begin{enumerate}

  \item We first modeled the proportion of $G_{fir}$ ($P_{Gfir} = \frac{G_{fir}}{G_{fir} + G_{spruce}}$) at the PROTEST plots $i$ from environmental variables and LIDAR-derived measures, using a beta regression with a logit link. The model is based on the following initial structure:

  \begin{multline}\label{gfir-spruce}
    \widehat{P_{Gfir,i}} = logit(a + b_1 alti_i + b_2 slope_i + b_3 rum_i + b_4 pH_i + b_5 expoNS_i + b_6 expoEW_i + \\ b_7 dg_i + b_8 g_i + b_9 greco_i + b_{10} hyrdo_i + b_{11} carbo_i)
  \end{multline}

  \begin{equation}\label{}
    P_{Gfir,i} \sim \mathcal{B}e (\widehat{P_{Gfir,i}}\phi, (1-\widehat{P_{Gfir,i}})\phi)
  \end{equation}

  \noindent where $a$, $b_{1-11}$ and $\phi$ are parameters to estmiate. We applied the same variable selection procedure as the one we applied to the site index model (see final model in SI). Here, we worked with PROTEST plots where fir and spruce represented more than 75\% of the total basal area of trees (saplings are not taken into account when selecting the PROTEST plots since we do not know which species they belong to). We assigned the basal area of saplings and non-target species to $G_{fir}$ and $G_{spruce}$ depending on the relative proportion of fir and spruce.

  \item We then assigned $P_{Gfir}$ values to each forest plot $p$ consituted of a fir-spruce mixture over the study area by drawing random numbers from the Beta distribution $\mathcal{B}e (\widehat{P_{Gfir,p}}\phi, (1-\widehat{P_{Gfir,p}})\phi)$, $\widehat{P_{Gfir,p}}$ being the $P_{Gfir}$ predicted by the model at plot $p$. $G_{fir}$ and $G_{spruce}$ could then be calculated from $P_{Gfir}$ and $G_t$.

\end{enumerate}

%%%%%%%%%%%%%%%%%%%%%%%%%%%%%%%

\subsection*{Species quadratic diameter}\label{Dg}

We assigned species quadratic diameter ($Dg_{sp}$) to each forest plot in different ways depending on the stand composition. In the case of pure stands, we directly used the total quadratic diameter ($Dg_t$) calculated from LIDAR data. We developed a three-step procedure for mixed stands:

\begin{enumerate}

  \item We first modeled the $Dg_{sp}$ ratio of the mixed species ($sp1$ and $sp2$; $R_{Dg} = \frac{Dg_{sp1}}{Dg_{sp2}}$) at the PROTEST plots $i$, and that, separately for each mixture (beech-fir, beech-spruce and fir-spruce), from environmental variables and LIDAR-derived measures, using Generalized linear models with a Gamma ($\Gamma$) probability distribution. The three mixture-specific models are based on the following structure:

  \begin{multline}\label{gfir-spruce}
    \widehat{R_{Dg, i}} = a + b_1 alti_i + b_2 slope_i + b_3 rum_i + b_4 pH_i + b_5 expoNS_i + b_6 expoEW_i + \\ b_7 dg_i + b_8 g_i + b_9 greco_i + b_{10} hyrdo_i + b_{11} carbo_i
  \end{multline}

  \begin{equation}\label{si}
    R_{Dg, i} \sim \Gamma(\alpha, \frac{\alpha}{\widehat{R_{Dg, i}}})
  \end{equation}

  where $a$, $b_{1-11}$ and $\alpha$ are parameters to estmiate. We applied the same variable selection procedure as the one we applied to the site index model (see the three mixture-specific final models in SI). Here, we worked with PROTEST plots where the two mixed species ($sp1$ and $sp2$) represented more than 75\% of the total basal area of trees (saplings are not taken into account when selecting the PROTEST plots since we do not know which species they belong to).

  \item We then assign $R_{Dg}$, the each forest plot $p$ constituted of a mixed stand over the study area by drawing random numbers from the Gamma distribution $\Gamma(\alpha, \frac{\alpha}{\widehat{R_{Dg, p}}})$, $\widehat{R_{Dg, p}}$ being the $R_{Dg}$ predicted by the models at plot $p$.

  \item We then calculated the species quadratic diameter using two equations linking species basal area to species quadratic diameter:

  \begin{equation}\label{Gsp}
    G_{sp} = \frac{n_{sp}\pi Dg_{sp}^2}{4}
  \end{equation}

  where $G_{sp}$, $D_{sp}$ and $n_{sp}$ are the basal area, the quadratic diameter and the number of individuals of a species, respectively; and

  \begin{equation}\label{}
    Dg_t^2 = \frac{n_{sp1}Dg_{sp1}^2 + n_{sp2}Dg_{sp2}^2}{n_{sp1} + n_{sp2}}
  \end{equation}

  where $Dg_t$ is the total quadratic diamater of a stand; $Dg_{sp1}$ and $Dg_{sp2}$, the quadratic diameters of species $sp1$ and $sp2$; and $n_{sp1}$ and $n_{sp2}$, the number of individuals of species $sp1$ and $sp2$. $Dg_{sp1}$ could therefore be expressed as:

  \begin{equation}\label{}
    Dg_{sp1} = Dg_t\sqrt{\frac{G_{sp1} + R_{Dg}^2G_{sp2}}{G_{sp1} + G_{sp2}}}
  \end{equation}

  where $G_{sp1}$ and $G_{sp2}$ are the species basal area (see proof in SI). We finally calculated $Dg_{sp2}$ as follows:

  \begin{equation}\label{}
    Dg_{sp2} = \frac{Dg_{sp1}}{R_{Dg}}
  \end{equation}

\end{enumerate}

%%%%%%%%%%%%%%%%%%%%%%%%%%%%%%%

\subsection*{Number of trees per species}

\noindent We assigned a number of trees to each species ($n_{sp}$) and on each forest plot by transforming equation \ref{Gsp} as follows:

\begin{equation}\label{}
  n_{sp} = \frac{4G_{sp}}{\pi Dg_{sp}^2}
\end{equation}

\noindent where $G_{sp}$ and $Dg_{sp}$ are the species basal area and quadratic diameter obtained from the procedures detailed above.

%%%%%%%%%%%%%%%%%%%%%%%%%%%%%%%%%%%%%%%%%%%%%%%%%%%%%%%%%%%%%%%%%%%%%%%%%%%%%%%%%
\section*{Results}

%%%%%%%%%%%%%%%%%%%%%%%%%%%%%%%%%%%%%%%%%%%%%%%%%%%%%%%%%%%%%%%%%%%%%%%%%%%%%%%%%
\section*{Discussion}


%%%%%%%%%%%%%%%%%%%%%%%%%%%%%%%%%%%%%%%%%%%%%%%%%%%%%%%%%%%%%%%%%%%%%%%%%%%%%%%%%
\section*{Authors' Contributions}

\section*{Data Accessibility}

\bibliographystyle{abbrvnat}
\bibliography{sample}

\clearpage

%%%%%%%%%%%%%%%%%%%%%%%%%%%%%%%%%%%%%%%%%%%%%%%%%%%%%%%%%%%%%%%%%%%%%%%%%%%%%%%%%%
\section*{SUPPORTING INFORMATION}


\clearpage

\noindent Proof: calculating species quadratic diameters from a set of three equations

\noindent The three starting equations are as follows:

\begin{equation}\label{one}\tag{a}
  \bullet G_{sp} = \frac{n_{sp}\pi Dg_{sp}^2}{4}
\end{equation}

\noindent where $G_{sp}$, $D_{sp}$ and $n_{sp}$ are the basal area, the quadratic diameter and the number of individuals of a species, respectively.

\begin{equation}\label{two}\tag{b}
  \bullet Dg_t^2 = \frac{n_iDg_i^2 + n_jDg_j^2}{n_i + n_j}
\end{equation}

\noindent where $Dg_t$ is the total quadratic diamater of a stand; $Dg_i$ and $Dg_j$, the quadratic diameters of species $i$ and $j$; and $n_i$ and $n_j$, the number of individuals of species $i$ and $j$.

\begin{equation}\label{three}\tag{c}
  \bullet \frac{Dg_i}{Dg_j} = r_{Dg}
\end{equation}

\noindent where $r_{Dg}$ is calculated by adding to the models predictions (models described at eqn \ref{DgModel}) a random number drawn from the normal distribution $\mathcal{N} (0, \sigma_\varepsilon)$, with $\sigma_\varepsilon$ being the standard deviation of the models residuals (see the \textit{Species quadratic diameter} section).

\hfill

\noindent Using equation \ref{one}, $Dg_t^2$ in equation \ref{two} can be expressed as:

\begin{equation*}\label{}
  Dg_t^2 = \frac{\frac{4}{\pi}G_i + \frac{4}{\pi}G_j}{\frac{4G_i}{\pi Dg_i^2} + \frac{4G_j}{\pi Dg_j^2}}
\end{equation*}

\begin{equation*}\label{}
  Dg_t^2 = \frac{G_i + G_j} {\frac{G_i}{Dg_i^2} + \frac{G_j}{Dg_j^2}}
\end{equation*}

\begin{equation*}\label{}
  Dg_t^2(\frac{G_i}{Dg_i^2} + \frac{G_j}{Dg_j^2})= G_i + G_j
\end{equation*}

\begin{equation*}\label{}
  Dg_t^2(G_i + \frac{Dg_i^2}{Dg_j^2}G_j^2)= Dg_i^2(G_i + G_j)
\end{equation*}

\noindent Using equation \ref{three}, $Dg_i^2 / Dg_j^2$ can be replaced by $r_{Dg}^2$:

\begin{equation*}\label{}
  Dg_t^2(G_i + r_{Dg}^2G_j^2)= Dg_i^2(G_i + G_j)
\end{equation*}

\noindent hence:

\begin{equation*}\label{}
  Dg_i = Dg_t\sqrt{\frac{G_i + r_{Dg}^2G_j}{G_i + G_j}}
\end{equation*}



\end{document}
