\documentclass[a4paper]{article}

%% Language and font encodings
\usepackage[english]{babel}
\usepackage[utf8]{inputenc}
\usepackage{times}

%% Sets page size and margins
\usepackage[a4paper,top=3cm,bottom=2cm,left=3cm,right=3cm,marginparwidth=1.75cm]{geometry}

%% Useful packages
\usepackage{amsmath}
\usepackage{graphicx}
\usepackage{xcolor}

%% multiline equations
\usepackage{amsmath}

%% line number
\usepackage[left]{lineno}
\linenumbers

%% double space (with \doublespacing)
\usepackage{setspace}
\doublespacing

%% citation / bibliography
\usepackage[round,authoryear]{natbib}


\usepackage{authblk}

\title{PROTEST}

\author[1]{Raphaël Aussenac}
\author[1]{Patrick Vallet}
\author[1]{Jean-Matthieu Monnet}
\author[2]{Thomas Carrette}
\author[3]{Pierre Paccard}
\author[4]{Arnaud Sergent}
\author[5]{Catherine Riond}
\author[6]{Raphaël Bondu}
\author[1]{Vanneck Nzeta Kenne}

\affil[1]{IRSTEA grenoble}
\affil[2]{FCBA grenoble}
\affil[3]{PNR Bauges}
\affil[4]{IRSTEA Bordeaux}
\affil[5]{ONF}
\affil[6]{???}


\date{}

\begin{document}
\maketitle

\section*{Abstract}

\section*{Journal}
\begin{itemize}
    \item Environmental Modelling \& Software (suggestion Patrick)
    \item Agricultural and Forest Meteorology (Coupling transversal and longitudinal models to better predict Quercus petraea and Pinus sylvestris stand growth under climate change)
    \item Forest Ecology and Management
    \item Annals of Forest Science (Capsis: an open software framework and communityfor forest growth modelling)
    \item Journal of Environmental Management
    \item Methods in ecology and evolution (papier sur méthode d'initialisation)
\end{itemize}

%%%%%%%%%%%%%%%%%%%%%%%%%%%%%%%%%%%%%%%%%%%%%%%%%%%%%%%%%%%%%%%%%%%%%%%%%%%%%%%%%
\section*{Introduction}

\citep{cardinale2012biodiversity}

%%%%%%%%%%%%%%%%%%%%%%%%%%%%%%%%%%%%%%%%%%%%%%%%%%%%%%%%%%%%%%%%%%%%%%%%%%%%%%%%%
\section*{Materials and Methods}

%%%%%%%%%%%%%%%%%%%%%%%%%%%%%%%

\subsection*{Geological classification - Raphaël Bondu}

Description de la variable “code carbonate”:

Cette variable décrit la teneur en minéraux carbonatés de la roche-mère suivant 4 catégories :
0  Pas à peu de carbonates (ex. schiste)
1  Présence modérée de carbonates (ex. marne)
2  Riche en carbonates (ex. calcaire argileux)
3  Carbonates quasi-purs (ex. calcaire lithographique)
Lorsque les ensembles géologiques regroupent des formations variées du point de vue lithologique, la valeur assignée a tenté de tenir compte de l’épaisseur de chacune des formations.
Cette classification est basée sur les informations qualitatives tirées des documents du Bureau de Recherche Géologique et Minière (BRGM) suivants :
Doudoux B., Barféty J.C., Vivier G., Carfantan J.C., Nicoud G., Tardy M., avec la collaboration de Monjuvent G., Debon F., Menot R.-P., Aprahamian J. (1999) - Notice explicative, Carte géol. France (1/50 000), feuille Albertville (726). Orléans : BRGM, 119 p. Carte géologique par B. Doudoux, J.-C. Barféty, G. Vivier, J.-C. Carfantan, G. Nicoud, B. Colletta, M. Tardy (1999).
F. Cagnard (2008) – Carte géologique harmonisée du département de la Haute-Savoie. BRGM/RP-5 59 30 -FR, 329 p.,7 fig., 2 tab., 3 pl. Hors-texte.

\noindent Description de la variable “code hydro”:

Cette variable décrit la perméabilité potentielle de la roche-mère en fonction de sa nature minéralogique suivant 4 catégories :
0  Fortement perméable (ex. calcaire massif karstifié)
1  Perméable (ex. calcaire argileux)
2  Peu perméable (ex. alternance marne et calcaire)
3  Imperméable ou quasi-imperméable (ex. argile)
Cette classification est basée sur les informations qualitatives tirées des documents du Bureau de Recherche Géologique et Minière (BRGM) suivants :
Doudoux B., Barféty J.C., Vivier G., Carfantan J.C., Nicoud G., Tardy M., avec la collaboration de Monjuvent G., Debon F., Menot R.-P., Aprahamian J. (1999) - Notice explicative, Carte géol. France (1/50 000), feuille Albertville (726). Orléans : BRGM, 119 p. Carte géologique par B. Doudoux, J.-C. Barféty, G. Vivier, J.-C. Carfantan, G. Nicoud, B. Colletta, M. Tardy (1999).
F. Cagnard (2008) – Carte géologique harmonisée du département de la Haute-Savoie. BRGM/RP-5 59 30 -FR, 329 p.,7 fig., 2 tab., 3 pl. hors-texte.

%%%%%%%%%%%%%%%%%%%%%%%%%%%%%%%

\subsection*{Landscape initialisation}
To run SIMMEM simulations, the following variables must be available for each forest plot:
\begin{itemize}
    \item site index
    \item composition
    \item species basal area
    \item number of trees per species (species quadratic diameter must be calculated for that)
\end{itemize}

%%%%%%%%%%%%%%%%%%%%%%%%%%%%%%%

\subsection*{Site index}

\noindent We assigned a species-specific site index ($si$) to all forest plots over the study area following a two-step procedure:

\begin{enumerate}
    \item Because not all the variables required to calculate $si$ were available at all locations over the study area, we first modeled the available $si$ at the NFI plots from available environmental variables and LIDAR-derived measures as follows:

    \begin{equation}\label{si}
      si = \alpha + \beta_1 alti + \beta_2 pH +\beta_3 rum +\beta_4 slope +\beta_5 rocheCalc+...+\varepsilon
    \end{equation}

    where $\alpha$ and $\beta_{1-X}$ are parameters to estmiate and $\varepsilon$, the models residuals. We then applied to the models the following variable selection procedure:

    \begin{enumerate}

        \item We first removed (one by one) the variables whose effect were non-significant starting with the least significant ones untill all variables were significant.

        \item We then added a quadratic effect to all remaining variables (e.g. $\beta_1alti + \beta_1alti^2$).

        \item Finally, we removed (one by one) the variables whose effect became non-significant following the addition of the quadratic effects starting with the least significant ones untill all variables were significant (see the four final species-specific models in SI).

  \end{enumerate}

    \item We then calculated $si$ for all forest plots over the study area by adding to the models predictions a random number drawn from the normal distribution $\mathcal{N} (0, \sigma_\varepsilon)$, with $\sigma_\varepsilon$ being the standard deviation of the models residuals.

\end{enumerate}

%%%%%%%%%%%%%%%%%%%%%%%%%%%%%%%

\subsection*{Composition}

We assigned a composition to all forest plots over the study area following a four-step procedure:

\begin{enumerate}

    \item First we classified the PROTEST plots into 7 compositions (corresponding to the compositions that will be simulated): 4 pure compositions (beech, oak, fir, spruce; when these species represented more than 75\% of the plots total basal area) and 3 mixed compositions (beech-fir, beech-spruce, fir-spruce; when these species were the 2 most abundant on the plots and represented (together) more than 75\% of the plots total basal area). However, because other species than our 4 target species were present on the study area (ref composition / species abundance?), we previously grouped some species together as follows:

    \begin{enumerate}

        \item When non-target coniferous species (i.e. species other than fir and spruce) were present on a PROTEST plot, their basal area were assigned to fir and/or spruce (depending on the relative proportion of the basal area of these two species) if they were present on the plot.

        \item Similarly, when non-target deciduous species (i.e. species other than oak and beech) were present on a PROTEST plot, their basal area were assigned to oak and/or beech (depending on the relative proportion of the basal area of these two species) if they were present on the plot. However, there is one exception here: the basal area of \textit{Quercus robur} and unspecified oaks have always been assigned to \textit{Quercus petraea}.

    \end{enumerate}

    Plots whose composition was not among the 7 expected compositions were dropped at this stage.

    \item We then classified all the forest plots over the study area into 3 composition types: deciduous, coniferous or deciduous-coniferous mixture. For that we used the proportion of deciduous basal area ($P_{Gd}$) calculated from LIDAR data. Plots in which the deciduous basal area accounted for more than 75\% of the total basal area were classified as deciduous. Conversely, plots in which the deciduous basal area accounted for less than 25\% of the total basal area were classified as coniferous. The remaining plots were classified as deciduous-coniferous mixture.

    \item Thereafter, we retrieved the list of compositions observed of the PROTEST plots within each vegetation type (i.e. TFV type). This list could contain repetitions (because the same composition could be observed several times within a single TFV type). However, because 5 TFV types represented more than 96\% of the forest area and because the remaining 3\% belonged to 15 different TFV types, we previously grouped some TFV types together. The 15 under-represented TFV types have either been removed or integrated into one of the 5 main TFV types (see SI).

    \item Finally, to assign a composition to a forest plot located in a specific TFV type, we randomly selected a composition from the list of compositions associated to this TFV type while complying with the composition type allready assigned to the plot (i.e. deciduous, coniferous or deciduous-coniferous mixture). Thus plots classified as deciduous could be assigned 2 compositions: beech or oak; plots classified as coniferous could be assigned 3 compositions: fir, spruce or fir-spruce mixture; and plots classified as deciduous-coniferous mixture could be assigned 2 compositions: beech-fir or beech-spruce.

\end{enumerate}

%%%%%%%%%%%%%%%%%%%%%%%%%%%%%%%

\subsection*{Species basal area}

\noindent We assigned species basal area values ($G_{sp}$) to each forest plot using the stand total basal area ($G_t$) and the proportion of deciduous basal area ($P_{Gd}$) calculated from LIDAR data. $G_{sp}$ could directly be derived from these measures for pure stands and deciduous-coniferous mixtures (i.e. beech-fir and beech-spruce stands). We developed a two-step procedure to assign $G_{sp}$ to conifer mixtures (i.e. fir-spruce mixtures):

\begin{enumerate}

  \item We first modeled the proportion of $G_{fir}$ in the PROTEST plots from environmental variables and LIDAR-derived measures (and modeled site index???) as follows:

  \begin{equation}\label{gfir-spruce}
    \frac{G_{fir}}{G_{fir} + G_{spruce}} = \alpha + \beta_1 alti + \beta_2 pH +\beta_3 rum +\beta_4 slope +\beta_5 rocheCalc+...+ \varepsilon
  \end{equation}

  \noindent where $\alpha$ and $\beta_{1-X}$ are parameters to estmiate and $\varepsilon$, the model residuals. We applied the same variable selection procedure as the one we applied to the site index model (see final model in SI). Here, we used $G_{fir}$ and $G_{spruce}$ obtained from "true fir-spruce mixtures" and after having assigned the basal area of non-target species to the target species (see the \textit{composition initialisation} section). "True mixtures" were PROTEST plots where the two mixed species represented more than 75\% of the total basal area before the grouping of species.

  \item We then calculated $P_{Gfir}$, the proportion of $G_{fir}$ for all conifer mixtures over the study area by adding to the model predictions a random number drawn from the normal distribution $\mathcal{N} (0, \sigma_\varepsilon)$, with $\sigma_\varepsilon$ being the standard deviation of the model residuals. We made sure that $0.25 < P_{Gfir} < 0.75$ to comply with the definition of mixture that we use in this study. Another random number was drawn and added to the model prediction until this condition was met. $G_{fir}$ and $G_{spruce}$ could then be calculated from $P_{Gfir}$ and $G_t$.

\end{enumerate}

%%%%%%%%%%%%%%%%%%%%%%%%%%%%%%%

\subsection*{Species quadratic diameter}\label{Dg}

We assigned species quadratic diameter ($Dg_{sp}$) to each forest plot in different ways depending on the stand composition. In the case of pure stands, we directly used the total quadratic diameter ($Dg_t$) calculated from LIDAR data. We developed a three-step procedure for mixed stands:

\begin{enumerate}

  \item We first modeled the species quadratic diameters ratio in the PROTEST plots (separately for each mixture, i.e. beech-fir, beech-spruce and fir-spruce) from environmental variables and LIDAR-derived measures (and modeled site index???) as follows:

  \begin{equation}\label{DgModel}
    \frac{Dg_i}{Dg_j} = \alpha + \beta_1 alti + \beta_2 pH +\beta_3 rum +\beta_4 slope +\beta_5 rocheCalc+...+\varepsilon
  \end{equation}

  where $Dg_i$ and $Dg_j$ are the quadratic diameters of species $i$ and $j$; $\alpha$ and $\beta_{1-X}$, parameters to estmiate; and $\varepsilon$, the model residuals. Here, we used $Dg_i$ and $Dg_j$ obtained from "true mixtures", i.e. PROTEST plots where the two mixed species represented more than 75\% of the total basal area, and this, before the grouping of species. We applied the same variable selection procedure as the one we applied to the site index model (see the three mixture-specific final models in SI).

  \item We then calculated $r_{Dg}$, the $Dg_i / Dg_j$ ratios for all mixed stand over the study area by adding to the models predictions a random number drawn from the normal distribution $\mathcal{N} (0, \sigma_\varepsilon)$, with $\sigma_\varepsilon$ being the standard deviation of the models residuals. We made sure that $r_{Dg}$ falled in the range of the $Dg_i / Dg_j$ ratio values directly measured in the PROTEST "true mixtures" to avoid producing stands with excessively unbalanced diameters for both species. Another random number was drawn and added to the models predictions until this condition was met.

  \item We then calculated the species quadratic diameter using two equations linking species basal area to species quadratic diameter:

  \begin{equation}\label{Gsp}
    G_{sp} = \frac{n_{sp}\pi Dg_{sp}^2}{4}
  \end{equation}

  where $G_{sp}$, $D_{sp}$ and $n_{sp}$ are the basal area, the quadratic diameter and the number of individuals of a species, respectively; and

  \begin{equation}\label{}
    Dg_t^2 = \frac{n_iDg_i^2 + n_jDg_j^2}{n_i + n_j}
  \end{equation}

  where $Dg_t$ is the total quadratic diamater of a stand; $Dg_i$ and $Dg_j$, the quadratic diameters of species $i$ and $j$; and $n_i$ and $n_j$, the number of individuals of species $i$ and $j$. $Dg_i$ could therefore be expressed as:

  \begin{equation}\label{}
    Dg_i = Dg_t\sqrt{\frac{G_i + r_{Dg}^2G_j}{G_i + G_j}}
  \end{equation}

  where $G_i$ and $G_j$ are the species basal area (see proof in SI). We finally calculated $Dg_j$ as follows:

  \begin{equation}\label{}
    Dg_j = \frac{Dg_i}{r_{Dg}}
  \end{equation}

\end{enumerate}

%%%%%%%%%%%%%%%%%%%%%%%%%%%%%%%

\subsection*{Number of trees per species}

\noindent We assigned a number of trees to each species ($n_{sp}$) and on each forest plot by transforming equation \ref{Gsp} as follows:

\begin{equation}\label{}
  n_{sp} = \frac{4G_{sp}}{\pi Dg_{sp}^2}
\end{equation}

\noindent where $G_{sp}$ and $Dg_{sp}$ are the species basal area and quadratic diameter obtained from the procedures detailed above.

%%%%%%%%%%%%%%%%%%%%%%%%%%%%%%%%%%%%%%%%%%%%%%%%%%%%%%%%%%%%%%%%%%%%%%%%%%%%%%%%%
\section*{Results}

%%%%%%%%%%%%%%%%%%%%%%%%%%%%%%%%%%%%%%%%%%%%%%%%%%%%%%%%%%%%%%%%%%%%%%%%%%%%%%%%%
\section*{Discussion}


%%%%%%%%%%%%%%%%%%%%%%%%%%%%%%%%%%%%%%%%%%%%%%%%%%%%%%%%%%%%%%%%%%%%%%%%%%%%%%%%%
\section*{Authors' Contributions}

\section*{Data Accessibility}

\bibliographystyle{abbrvnat}
\bibliography{sample}

\clearpage

%%%%%%%%%%%%%%%%%%%%%%%%%%%%%%%%%%%%%%%%%%%%%%%%%%%%%%%%%%%%%%%%%%%%%%%%%%%%%%%%%%
\section*{SUPPORTING INFORMATION}


\clearpage

\noindent Proof: calculating species quadratic diameters from a set of three equations

\noindent The three starting equations are as follows:

\begin{equation}\label{one}\tag{a}
  \bullet G_{sp} = \frac{n_{sp}\pi Dg_{sp}^2}{4}
\end{equation}

\noindent where $G_{sp}$, $D_{sp}$ and $n_{sp}$ are the basal area, the quadratic diameter and the number of individuals of a species, respectively.

\begin{equation}\label{two}\tag{b}
  \bullet Dg_t^2 = \frac{n_iDg_i^2 + n_jDg_j^2}{n_i + n_j}
\end{equation}

\noindent where $Dg_t$ is the total quadratic diamater of a stand; $Dg_i$ and $Dg_j$, the quadratic diameters of species $i$ and $j$; and $n_i$ and $n_j$, the number of individuals of species $i$ and $j$.

\begin{equation}\label{three}\tag{c}
  \bullet \frac{Dg_i}{Dg_j} = r_{Dg}
\end{equation}

\noindent where $r_{Dg}$ is calculated by adding to the models predictions (models described at eqn \ref{DgModel}) a random number drawn from the normal distribution $\mathcal{N} (0, \sigma_\varepsilon)$, with $\sigma_\varepsilon$ being the standard deviation of the models residuals (see the \textit{Species quadratic diameter} section).

\hfill

\noindent Using equation \ref{one}, $Dg_t^2$ in equation \ref{two} can be expressed as:

\begin{equation*}\label{}
  Dg_t^2 = \frac{\frac{4}{\pi}G_i + \frac{4}{\pi}G_j}{\frac{4G_i}{\pi Dg_i^2} + \frac{4G_j}{\pi Dg_j^2}}
\end{equation*}

\begin{equation*}\label{}
  Dg_t^2 = \frac{G_i + G_j} {\frac{G_i}{Dg_i^2} + \frac{G_j}{Dg_j^2}}
\end{equation*}

\begin{equation*}\label{}
  Dg_t^2(\frac{G_i}{Dg_i^2} + \frac{G_j}{Dg_j^2})= G_i + G_j
\end{equation*}

\begin{equation*}\label{}
  Dg_t^2(G_i + \frac{Dg_i^2}{Dg_j^2}G_j^2)= Dg_i^2(G_i + G_j)
\end{equation*}

\noindent Using equation \ref{three}, $Dg_i^2 / Dg_j^2$ can be replaced by $r_{Dg}^2$:

\begin{equation*}\label{}
  Dg_t^2(G_i + r_{Dg}^2G_j^2)= Dg_i^2(G_i + G_j)
\end{equation*}

\noindent hence:

\begin{equation*}\label{}
  Dg_i = Dg_t\sqrt{\frac{G_i + r_{Dg}^2G_j}{G_i + G_j}}
\end{equation*}



\end{document}
